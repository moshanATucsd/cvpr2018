\documentclass[10pt,twocolumn,letterpaper]{article}

\usepackage{cvpr}
\usepackage{times}
\usepackage{epsfig}
\usepackage{graphicx}
\usepackage{amsmath}
\usepackage{amssymb}

% Include other packages here, before hyperref.

% If you comment hyperref and then uncomment it, you should delete
% egpaper.aux before re-running latex.  (Or just hit 'q' on the first latex
% run, let it finish, and you should be clear).
\usepackage[breaklinks=true,bookmarks=false]{hyperref}

\cvprfinalcopy % *** Uncomment this line for the final submission

\def\cvprPaperID{****} % *** Enter the CVPR Paper ID here
\def\httilde{\mbox{\tt\raisebox{-.5ex}{\symbol{126}}}}

% Pages are numbered in submission mode, and unnumbered in camera-ready
%\ifcvprfinal\pagestyle{empty}\fi
\setcounter{page}{4321}
\begin{document}

%%%%%%%%% TITLE
\title{Spatio-Temporal reconstruction of Rigid objects from multiple cameras}


\maketitle
%\thispagestyle{empty}

%%%%%%%%% ABSTRACT
\begin{abstract}
Multi camera based understanding of dynamic scenes from unsynchronized images is a challening problem in reconstruction. We propose a solution to the problem of 3D reconstruction of multiple moving rigid objects from unsynchronized multiple cameras. Most of the reconstruction piplines leverage reconstruction using reprojection error of features, we propose a novel semantic features pipeline and object based feature tracking for reconstruction of dynamic objects. We leverage multiple rigid object constraints in a bundle adjustment framework to improve the reconstuction. \end{abstract}

\section{METHOD}
We Follow the basic pipeline of reconstcruion i.e. feature detection and tracking, correspodence of the keypoint across views and then bundle adjustment for accurate reconstruction of the 3d points. In each of the segments of the pipeline we incorporate geometric and semantic information of the objects for better reconstruction. We propose instance keypoint segmentation for better feature detection of important points on the rigid object.    

%-------------------------------------------------------------------------
\subsection{Instance Aware Keypoint detection}
Given a images with multiple moving objects. The amount of objects which can move can be sorted into a small set of categories like cars and humans. We propose a instance keypoint detection on these categories for improving the reconstruction of these objects. We leverage deep learning framwork for fast and accurate detection of semantic features in images. We pass the input image through a detection pipeline and each detected bounding box is passed through a object specific keypoint detection pipeline for computation of important parts in the image. 


\end{document}
